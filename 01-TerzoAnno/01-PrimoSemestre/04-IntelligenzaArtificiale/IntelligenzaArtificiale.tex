\documentclass[a4paper, 12pt]{book}
\usepackage[margin=2.5cm]{geometry}
\usepackage[italian]{babel}
%\usepackage{wrapfig}
\usepackage{graphicx}
\usepackage{hyperref}
\usepackage{amsmath}
\setlength{\parindent}{0pt}
\usepackage{parskip}
\usepackage[most]{tcolorbox}
\usepackage{soul}
\usepackage{xcolor}
\sethlcolor{yellow}
\usepackage{array}
\usepackage{booktabs}
\usepackage{tikz}
\usetikzlibrary{positioning}
\usepackage{longtable}
\usepackage{array}
\usepackage{fancyhdr}
\setlength{\headheight}{15pt}
\pagestyle{fancy}
\fancyhf{} % pulisce header e footer
\fancyhead[R]{\thepage} % numero pagina a destra nell'header
\renewcommand{\headrulewidth}{0pt} % (opzionale) elimina la riga orizzontale sotto l'header
\fancypagestyle{plain}{%
  \fancyhf{} % pulisce header e footer
  \fancyhead[R]{\thepage} % numero sempre in alto a destra
  \renewcommand{\headrulewidth}{0pt} % opzionale: niente linea
}
\setcounter{tocdepth}{3}

\title{\textbf{Intelligenza artificiale}}
\author{Mattia Nicolis}
\date{A.A. 2025-26}

\begin{document}

    \maketitle

    \tableofcontents
    \markboth{}{}

    %-------MODULO DI TEORIA------------------------------------
    %-------Prof: Alessandro Farinell---------------------------
    %-------Lez: mercoledì (8:30-11:30) [Aula Delta - Cav.3]----
    %-------Libro: Artificial Intelligence: a modern approach---
    \chapter*{Introduzione all'intelligenza artificiale}
    \addcontentsline{toc}{chapter}{Introduzione all'intelligenza artificiale}
    Il compito dell'\textbf{intelligenza artificiale} (o IA), non è solo di \textit{comprendere}, ma anche di \textit{costruire} entità intelligenti: macchine in grado di calcolare come agire in modo efficace e sicuro in un'ampia vairetà di situazioni nuove.

    In passato gli studiosi hanno diverse versioni di IA: alcuni hanno definito l'intelligenza in termini di fedeldeltà alla \textbf{prestazione umana}, mentre altri preferiscono una definizione formale di intelligenza cime \textbf{razionalità}.

    I metodi usati sono necessariamente diversi: l'approcio che persegue un'intelligenza simile a quella umana deve essere in pare una scienza empirica correlata alla pscicologia, richiedend osservazioni e ipotesi riguardo il comprtamneto umano e i processi di pensiero.

    Un approcio razionalista, invece, sfrutta una combinazione di matematica e ingegneria e si collega alla statistica, alla teoria del controllo e all'economia.

    \begin{tcolorbox}[
      colback=cyan!5!white,
      colframe=blue!50!black,
      title=\textbf{Test di Turing},
      coltitle=white,
      fonttitle=\bfseries,
      arc=3mm,
      boxrule=0.5pt,
      enhanced,
      breakable
    ]
      Il \textbf{test di Turing}, prosto da Alan Turing nel 1950, è quella di comprendere se "\textit{una macchina è in gardo di pensare?}".

      Un computer supererà il test se un esaminatore umano, dopo aver posto alcune omande in forma scritta, non sarà in grado d capire se le risposte provengono da una persona o no.
      
      Il computer avrebbe bisogno delle seguenti capacità:
      \begin{itemize}
        \item \textbf{interpretazione del linguaggio naturale} per comunicare con successo nel linguaggio umano
        \item \textbf{rappresentazione della conoscenza} per memorizzare quello che conosce o sente
        \item \textbf{\textbf{ragionamento automatico}} per rispondere alle domande e trarre nuove conclusioni
        \item \textbf{apprendimento automatico} (\textit{machine learning}) per adattarsi a nuove circostanza, individuare ed estrapolare schemi
      \end{itemize}

      Tuttavia, esiste anche un cosidetto \textbf{test di Turing totale} che richiede ò'interazione on oggetti e persone nel mondo reale.

      Per superare il test di Turing totale, un robot necessiterà anche di:
      \begin{itemize}
        \item \textbf{visione artificiale} e riconosciemtno vocale per percepire il mondo
        \item \textbf{robotica} per manipolare gli oggetti e spostarsi fisicamente
      \end{itemize}
    \end{tcolorbox}

    Un \textbf{agente} è semplicemente qualcosa che agisce, che fa qualcosa.

    Tuttavia, si suppone che gli agenti artificali faccinao di più: operare autonomamente, essere in grado di percepire l'ambiente, persistere in un'attività per un luogo arco di tempo, adattersi ai cambiamenti e creare e perseguire degli obiettivi.

    Un \textbf{agente razionale} agisce in modo da ottenere il miglior risultato o, in condizioni di incertezza, il miglior risultato atteso.


    \chapter*{Agenti intelligenti}
    \addcontentsline{toc}{chapter}{Agenti intelligenti}
    Un \textbf{agente} è qualsiasi cosa possa essere vista come un \hl{sistema che percepisce il suo \textbf{ambiente} attraverso dei \textbf{sensori} e agisce su di esse mediante \textbf{attuatori}}.

    Un agente umano possiede come sensori \textit{occhi}, e \textit{altri organi}e può usare come attuatori \textit{mani}, \textit{gambe}, \textit{tratto vocale} ecc.

    Un agente robotico potrebbe avere telecamere e telemetri a infronatrsi per sensori e diversi motori per attuatori.

    %foto agente

    \begin{tcolorbox}[
      colback=cyan!5!white,
      colframe=blue!50!black,
      title=\textbf{Definizione formale di agente razionale},
      coltitle=white,
      fonttitle=\bfseries,
      arc=3mm,
      boxrule=0.5pt,
      enhanced,
      breakable
    ]
      Per ogni possibile seuqenza di percezioni, un agente razionale dovrebbe scegliere un'azione che massimizzi il valore atteso della sua misura, date le informazioni fornite dalla sequenza e da ogni ulteriore conoscenza dell'agente.
    \end{tcolorbox}


\end{document}
\documentclass[a4paper, 12pt]{book}
\usepackage[margin=2.5cm]{geometry}
\usepackage[italian]{babel}
%\usepackage{wrapfig}
\usepackage{graphicx}
\usepackage{hyperref}
\usepackage{amsmath}
\setlength{\parindent}{0pt}
\usepackage{parskip}
\usepackage{soul}
\usepackage{xcolor}
\sethlcolor{yellow}
\usepackage{array}
\usepackage{booktabs}
\usepackage{tikz}
\usetikzlibrary{positioning}
\usepackage{longtable}
\usepackage{array}
\usepackage{fancyhdr}
\setlength{\headheight}{15pt}
\pagestyle{fancy}
\fancyhf{} % pulisce header e footer
\fancyhead[R]{\thepage} % numero pagina a destra nell'header
\renewcommand{\headrulewidth}{0pt} % (opzionale) elimina la riga orizzontale sotto l'header
\fancypagestyle{plain}{%
  \fancyhf{} % pulisce header e footer
  \fancyhead[R]{\thepage} % numero sempre in alto a destra
  \renewcommand{\headrulewidth}{0pt} % opzionale: niente linea
}
\setcounter{tocdepth}{3}

\title{\textbf{Elaborazione di segnali e immagini}}
\author{Mattia Nicolis}
\date{A.A. 2025-26}

\begin{document}

    \maketitle

    \tableofcontents
    \markboth{}{}

    %-------MODULO DI TEORIA------------------------------------
    %-------Prof: Manuele Bicego--------------------------------
    %-------Lez: mercoledì (8:30-11:30) [Aula Delta - Cav.3]----
    \chapter*{Teoria}
    \addcontentsline{toc}{chapter}{Teoria}


    %-------MODULO DI LABORATORIO-------------------------------
    %-------Prof: Menegaz Gloria--------------------------------
    %-------Lez: giovedì (15:30-18:30) [Aula A - Cav.1]---------
    \chapter*{Laboratorio}
    \addcontentsline{toc}{chapter}{Laboratorio}

\end{document}
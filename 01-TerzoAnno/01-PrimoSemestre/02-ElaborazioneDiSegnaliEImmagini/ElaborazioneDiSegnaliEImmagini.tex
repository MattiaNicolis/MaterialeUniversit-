\documentclass[a4paper, 12pt]{book}
\usepackage[margin=2.5cm]{geometry}
\usepackage[italian]{babel}
%\usepackage{wrapfig}
\usepackage{graphicx}
\usepackage{hyperref}
\usepackage{amsmath}
\setlength{\parindent}{0pt}
\usepackage{parskip}
\usepackage[most]{tcolorbox}
\usepackage{soul}
\usepackage{xcolor}
\sethlcolor{yellow}
\usepackage{array}
\usepackage{booktabs}
\usepackage{tikz}
\usetikzlibrary{positioning}
\usepackage{longtable}
\usepackage{array}
\usepackage{fancyhdr}
\setlength{\headheight}{15pt}
\pagestyle{fancy}
\fancyhf{} % pulisce header e footer
\fancyhead[R]{\thepage} % numero pagina a destra nell'header
\renewcommand{\headrulewidth}{0pt} % (opzionale) elimina la riga orizzontale sotto l'header
\fancypagestyle{plain}{%
  \fancyhf{} % pulisce header e footer
  \fancyhead[R]{\thepage} % numero sempre in alto a destra
  \renewcommand{\headrulewidth}{0pt} % opzionale: niente linea
}
\setcounter{tocdepth}{3}

\title{\textbf{Elaborazione di segnali e immagini}}
\author{Mattia Nicolis}
\date{A.A. 2025-26}

\begin{document}

    \maketitle

    \tableofcontents
    \markboth{}{}

    %-------MODULO DI TEORIA------------------------------------
    %-------Prof: Manuele Bicego--------------------------------
    %-------Lez: mercoledì (8:30-11:30) [Aula Delta - Cav.3]----
    \chapter*{Teoria}
    \addcontentsline{toc}{chapter}{Teoria}

    \section*{Sistemi e segnali}
    \addcontentsline{toc}{section}{Sistemi e segnali}

    \begin{tcolorbox}[
      colback=cyan!5!white,
      colframe=blue!50!black,
      title=\textbf{Segnale},
      coltitle=white,
      fonttitle=\bfseries,
      arc=3mm,
      boxrule=0.5pt,
      enhanced,
      breakable
    ]
      Un \textbf{segnale} è un \hl{insieme di dati o informazioni} (segnale telefonico o televisivo ecc.).
    \end{tcolorbox}

    \vspace{2mm}

    I segnali sono funzioni della variabile indipendente \textit{tempo}.

    Un segnale può essere elaborato da un \textbf{sistema}, che li possono modificare o estrane informazioni aggiuntive.

    \vspace{2mm}

    \begin{tcolorbox}[
      colback=cyan!5!white,
      colframe=blue!50!black,
      title=\textbf{Sistema},
      coltitle=white,
      fonttitle=\bfseries,
      arc=3mm,
      boxrule=0.5pt,
      enhanced,
      breakable
    ]
      Un \textbf{sistema} è un'\hl{entità che elabora un insieme di segnali (\textit{input}) per produrre un altro insieme di segnali (\textit{output})}.
    \end{tcolorbox}


    \subsection*{Classificazione dei segnali}
    \addcontentsline{toc}{subsection}{Classificazione dei segnali}
    I segnali possono essere di4 tipologie:
    \begin{itemize}
      \item \textbf{segnali a tempo continuo} e \textbf{a tempo discreto}
      \item \textbf{segnali analogici} e \textbf{digitali}
      \item \textbf{segnali periodici} e \textbf{aperiodici}
      \item \textbf{segnali energetici} e \textbf{segnali di potenza}
      \item \textbf{segnali deterministici} e \textbf{probabilistci}
    \end{itemize}

    \subsubsection*{Segnali a tempo continuo e atempo discreto}
    \addcontentsline{toc}{subsubsection}{Segnali a tempo continuo e atempo discreto}







    %-------MODULO DI LABORATORIO-------------------------------
    %-------Prof: Menegaz Gloria--------------------------------
    %-------Lez: giovedì (15:30-18:30) [Aula A - Cav.1]---------
    \chapter*{Laboratorio}
    \addcontentsline{toc}{chapter}{Laboratorio}

\end{document}
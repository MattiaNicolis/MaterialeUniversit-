\documentclass[a4paper, 12pt]{book}
\usepackage[margin=2.5cm]{geometry}
\usepackage[italian]{babel}
%\usepackage{wrapfig}
\usepackage{graphicx}
\usepackage{hyperref}
\usepackage{amsmath}
\setlength{\parindent}{0pt}
\usepackage{parskip}
\usepackage{amsfonts}
\usepackage{soul}
\usepackage{xcolor}
\usepackage[most]{tcolorbox}
\sethlcolor{yellow}
\usepackage{array}
\usepackage{booktabs}
\usepackage{tikz}
\usetikzlibrary{positioning}
\usepackage{longtable}
\usepackage{array}
\usepackage{fancyhdr}
\setlength{\headheight}{15pt}
\pagestyle{fancy}
\fancyhf{} % pulisce header e footer
\fancyhead[R]{\thepage} % numero pagina a destra nell'header
\renewcommand{\headrulewidth}{0pt} % (opzionale) elimina la riga orizzontale sotto l'header
\fancypagestyle{plain}{%
  \fancyhf{} % pulisce header e footer
  \fancyhead[R]{\thepage} % numero sempre in alto a destra
  \renewcommand{\headrulewidth}{0pt} % opzionale: niente linea
}
\setcounter{tocdepth}{3}

\title{\textbf{Fondamenti dell'informatica}}
\author{Mattia Nicolis}
\date{A.A. 2025-26}

\begin{document}

    \maketitle

    \tableofcontents
    \markboth{}{}

    %-------MODULO DI TEORIA------------------------------------
    %-------Prof: Isabella Mastroeni----------------------------
    %-------Lez: lunedì (14:00-16:00) [Aula D - Cav.1]----------
    %-------Lez: martedì (8:30-10:30) [Aula A - Cav.1]----------

    %-------Ho unito slide della prof, appunti miei e il documento di Lorenzo
    \chapter*{Introduzione}
    \addcontentsline{toc}{chapter}{Introduzione}
    L'informatica si occupa di risolvere problemi.

    La soluzione di un problema è un \textbf{algoritmo}, ovvero una squenza finita di passi che, se seguiti correttamente, \hl{risolvono il problema}.

    Poichè ogni dato può essere rappresentato da numeri naturali, tramite una biizeione, è sufficiente considerare funzioni del tipo:
    \begin{equation*}
      f:\mathbb{N}\rightarrow\mathbb{N}
    \end{equation*}

    per modellare tutti i possibili problemi.

    \vspace{3mm}

    \begin{tcolorbox}[
      colback=cyan!5!white,
      colframe=blue!50!black,
      title=\textbf{Definizione - funzione calcolabile},
      coltitle=white,
      fonttitle=\bfseries,
      arc=3mm,
      boxrule=0.5pt,
      enhanced,
      breakable
    ]
      Una \textbf{funzione} $f$ si dice \textbf{calcolabile} se \hl{esiste un algoritmo che, dato un input $x$, restituisce $f(x)$}  
    \end{tcolorbox}

    \vspace{3mm}

    Nell'insieme delle funzioni $\mathbb{N}\rightarrow\mathbb{N}$, la parte delle funzioni calcolabili è strettamente contenuta.

    \vspace{3mm}

    \begin{tcolorbox}[
      colback=cyan!5!white,
      colframe=blue!50!black,
      title=\textbf{Definizione - alfabeto},
      coltitle=white,
      fonttitle=\bfseries,
      arc=3mm,
      boxrule=0.5pt,
      enhanced,
      breakable
    ]
      Un \textbf{alfabeto} $\Sigma$ è un'\hl{insieme finito di simboli}.
    \end{tcolorbox}

    \vspace{1.5mm}
    
    Esempio: consideriamo l'alfabeto $Sigma = \{0, 1\}$.

    Le stringhe componibili da $Sigma$ sono:
    \begin{equation*}
      \Sigma^{*} = \{\epsilon, 0, 1, 00, 01, 10, 11, 0000, \dots\}
    \end{equation*}

    $\Sigma^{*}$ = è detto \textbf{insieme delle sequenza di lunghezza finita sull'alfabeto}.

    La stringa $\epsilon$ è la stringa vuota.

    Poichè non vi è limite alla lunghezza delle stringhe, l'insieme $\Sigma^{*}$ e la sua cardinalità vale:
    \begin{equation*}
      |\Sigma^{*}| = |\mathbb{N}|
    \end{equation*}

    \vspace{3mm}

    \begin{tcolorbox}[
      colback=cyan!5!white,
      colframe=blue!50!black,
      title=\textbf{Definizione - algoritmo},
      coltitle=white,
      fonttitle=\bfseries,
      arc=3mm,
      boxrule=0.5pt,
      enhanced,
      breakable
    ]
      Un \textbf{algoritmo} $\Sigma$ è una \hl{seuqenza finita di simboli su un'alfabeto}.
    \end{tcolorbox}
    
    \section*{Induzione matematica}
    \addcontentsline{toc}{section}{induzione matematica}
    Per utilizzare il principio di induzione su un insieme, è necassario che su questo sussita una relazione d'ordine ben fondata.

    \begin{tcolorbox}[
      colback=cyan!5!white,
      colframe=blue!50!black,
      title=\textbf{Definizione - induzione matematica},
      coltitle=white,
      fonttitle=\bfseries,
      arc=3mm,
      boxrule=0.5pt,
      enhanced,
      breakable
    ]
      Sia $A$ un insieme con una relazione d'ordine $<$ ben fondata.

      \vspace{1mm}

      Sia $\pi$ una proprietà definita sugli elementi di $A$.

      \vspace{1mm}

      Allora vale la seguente corrisposndenza:
      \begin{equation*}
        \forall a \in A: \pi(a) \Leftrightarrow \forall a \in A: ((\forall b \in A.b < a: \pi(b))) \Rightarrow \pi(a)
      \end{equation*}
    \end{tcolorbox}


    \chapter*{Linguaggi regolari}
    \addcontentsline{toc}{chapter}{Linguaggi regolari}

    \begin{tcolorbox}[
      colback=cyan!5!white,
      colframe=blue!50!black,
      title=\textbf{Definizione - stringa},
      coltitle=white,
      fonttitle=\bfseries,
      arc=3mm,
      boxrule=0.5pt,
      enhanced,
      breakable
    ]
      Una \textbf{stringa}, o parola, è una \hl{sequenza finita di simboli su un alfabeto $\Sigma$}.
    \end{tcolorbox}

    \vspace{3mm}

    \begin{tcolorbox}[
      colback=cyan!5!white,
      colframe=blue!50!black,
      title=\textbf{Definizione - lunghezza di una stringa},
      coltitle=white,
      fonttitle=\bfseries,
      arc=3mm,
      boxrule=0.5pt,
      enhanced,
      breakable
    ]
      La lunghezza $|x|$ di una stringa $x$ è il numero di occorrenze di simboli al suo interno.
    \end{tcolorbox}

    Alcuni esempi di linguaggio sono $0$, dato che è sottoinsieme di qualsiasi insieme, e $\{\epsilon\}$, la stringa vuota.

    Quest'ultimo compone, appunto, il linguaggio $L$ che contiene solo la stringa vuota.

    \section*{Automi a stati finiti deterministici}
    \addcontentsline{toc}{section}{Automi a stati finiti deterministici}
    I linguaggi regolari sono tali per cui esiste un automa a stati finiti deterministico che li riconosce.

    Un automa è composto da un nastro di lunghezza illimitata, ma in sola lettura e dunque non possiede nessuna alcun tipo di memoria attiva.

    Il nastro viene letto mediante una testina e scorre in una sola direzione.

    L'automa possiede inoltre un'entità di controllo, descritta attraverso lo stato in cui si trova.

    Un programma in esecuzione sull'unità, infiene, si ossupa di stabilire le transizioni di stato.
\end{document}
\documentclass[a4paper, 12pt]{book}
\usepackage[margin=2.5cm]{geometry}
\usepackage[italian]{babel}
%\usepackage{wrapfig}
\usepackage{graphicx}
\usepackage{hyperref}
\usepackage{amsmath}
\setlength{\parindent}{0pt}
\usepackage{parskip}
\usepackage{soul}
\usepackage{xcolor}
\usepackage[most]{tcolorbox}
\usepackage{amsfonts}
\sethlcolor{yellow}
\usepackage{array}
\usepackage{booktabs}
\usepackage{tikz}
\usetikzlibrary{positioning}
\usepackage{longtable}
\usepackage{array}
\usepackage{fancyhdr}
\setlength{\headheight}{15pt}
\pagestyle{fancy}
\fancyhf{} % pulisce header e footer
\fancyhead[R]{\thepage} % numero pagina a destra nell'header
\renewcommand{\headrulewidth}{0pt} % (opzionale) elimina la riga orizzontale sotto l'header
\fancypagestyle{plain}{%
  \fancyhf{} % pulisce header e footer
  \fancyhead[R]{\thepage} % numero sempre in alto a destra
  \renewcommand{\headrulewidth}{0pt} % opzionale: niente linea
}
\setcounter{tocdepth}{3}

\title{\textbf{Logica}}
\author{Mattia Nicolis}
\date{A.A. 2025-26}

\begin{document}

    \maketitle

    \tableofcontents
    \markboth{}{}

    \chapter*{Introduzione alla logica}
    \addcontentsline{toc}{chapter}{Introduzione alla logica}
    La \textbf{logica} si occupa di \hl{formalizzare i ragionamenti matematici} utilizzando il linguaggio matematico.

    Ciò è dovuto al fatto che il linguaggio naturale è ambiguo per la descrizione di teorimi.

    La logica si divide in:
    \begin{itemize}
      \item \textbf{logica formale}: che a sua volta si divide in
      \begin{itemize}
        \item \textit{logica simbolica}
        \item \textit{Logica matematica}: studia la calcolabilità [studiata a fondamenti] 
      \end{itemize}
      \item \textbf{logica filosofica}: ha a che fare con il linguaggio naturale
    \end{itemize}

    \section*{Numeri naturali}
    \addcontentsline{toc}{section}{Numeri naturali}
    Dal punto di vista intuitivo i numeri sono un insieme infinito:
    \begin{equation*}
      \mathbb{N} = \{0, 1, 2, \dots\}
    \end{equation*}
    per i quali sono definite le operazioni standard aritmetiche elementari.

    Peano definì in maniera rigorosa i numeri naturali.

    \begin{tcolorbox}[
      colback=cyan!5!white,
      colframe=blue!50!black,
      title=\textbf{Definizione - numeri naturali},
      coltitle=white,
      fonttitle=\bfseries,
      arc=3mm,
      boxrule=0.5pt,
      enhanced,
      breakable
    ]
      I \textbf{numeri naturali} sono una tupla:
      \begin{equation*}
        \langle \mathbb{N}, 0, succ \rangle
      \end{equation*}

      Su quetsa tupla, sono definite 4 assiomi (teoremi che non si dimostrano):
      \begin{enumerate}
        \item $0 \in \mathbb{N}$
        \item $succ: \mathbb{N} \rightarrow \mathbb{N}$ (operazione unaria iniettiva)
        \item $0 \notin \mathrm{Im}(succ)$ (zero non è successore di nessuno)
        \item (\textbf{Principio di induzione}) per ogni $\mathbb{P} \subseteq \mathbb{N}$ valgono le seguenti proprietà:
        \begin{itemize}
          \item $0 \in \mathbb{P}$
          \item $\forall n \in \mathbb{N}, (n \in \mathbb{P} \rightarrow succ(n) \in \mathbb{P})$
        \end{itemize}
        $\Rightarrow \mathbb{P} = \mathbb{N}$
      \end{enumerate}
    \end{tcolorbox}
\end{document}